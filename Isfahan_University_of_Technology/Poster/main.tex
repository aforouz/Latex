\documentclass[12pt, a4paper]{article}

\usepackage[top=3cm, bottom=1.5cm, left=2.5cm, right=3cm]{geometry}

\usepackage{fancyhdr}
\usepackage{caption}
\usepackage{graphicx}
\usepackage{hyperref}
\usepackage{xepersian}
\settextfont[Scale=1.1]{Yas}
\setdigitfont[Scale=1.1]{Yas}
\defpersianfont\titr[Scale=1]{XB Titre}
\renewcommand{\baselinestretch}{1.3}

\begin{document}
\begin{titlepage}

\begin{center}

% ==========+==========+==========+==========+==========
\vspace*{-3cm}
\begin{figure}[h]
\centering
\includegraphics[height=30mm]{./logo.jpg}
\end{figure}
\vspace{-5mm}
%{\titr جلسه دفاع از پایان‌نامه کارشناسی ارشد}
% ==========+==========+==========+==========+==========
\\
\vspace{7mm}
{\Large
\textbf{کاربرد‌های توابع ماتریسی در جبر‌خطی عددی}
}
% ==========+==========+==========+==========+==========
\\
\vspace{5mm}
\textbf{سخنران:}
\textbf{علیرضا احمدی}
% ==========+==========+==========+==========+==========
\\
	\vspace{5mm}
\textbf{زمان:}
روز سه‌شنبه
\hspace{1cm}
تاریخ
26/01/1404
\hspace{1cm}
ساعت
30:16
% ==========+==========+==========+==========+==========
\\
\vspace{3mm}
\textbf{مکان:}
سالن خوارزمی دانشکده علوم ریاضی
% ==========+==========+==========+==========+==========
\\
\vspace{4mm}
\textbf{استاد راهنما:}
آقای دکتر رضا مختاری
% ==========+==========+==========+==========+==========
\\
% \vspace{3mm}
% \textbf{استاد مشاور:}
% خانم دکتر محدثه رمضانی
% % ==========+==========+==========+==========+==========
% \\
% \vspace{4mm}
% \textbf{هیئت داوران:}
% \begin{enumerate}
% \item
% آقای دکتر هادی روحانی (دانشگاه صنعتی مالک اشتر)
% \item
% خانم دکتر مریم محمدی (دانشگاه خوارزمی)
% \end{enumerate}
% ==========+==========+==========+==========+==========
\end{center}
% ==========+==========+==========+==========+==========
\noindent
\textbf{چکیده:}
در این ارائه سعی شده تا علاوه بر اشاره به تعاریف توابع ماتریسی و دلایل اهمیت آنها، روش‌های عددی و الگوریتمی آنها توسعه و بررسی شود.
کاربرد‌های توابع ماتریسی را می‌توان به دو دسته کلی تقسیم کرد.
دسته اول شامل کابردهای مستقیم و دسته دوم شامل کاربردهای توابع ماتریسی در روش‌های عددی برای حل دستگاها و روش‌های مربوطه است.
سعی شده که مطالب ارائه شده در راستای مفاهیم کلیدی جبرخطی عددی و بطور کلی روش‌ها و الگوریتم‌های عددی تشریح یابند.
برای مثال می‌توان به جواب‌های دستگاه‌های غیرخطی برداری و ماتریسی و به علاوه جواب‌های معادلات ریکاتی و همچنین معادله‌های سیلوستر اشاره کرد.
بطور ویژه برای حل یک دستگاه معادله دیفرانسیل برداری یا ماتریسی به دو مفهوم ریشه دوم یک ماتریس و همچنین تعریف توابع مثلثاتی برای یک ماتریس احتیاج پیدا خواهیم کرد. در اینجا ابهاماتی در مورد محدودیت ها و شروط مربوط به تعاریف مطرح می‌شود.
در این ارائه سعی شده است که علاوه بر پاسخ مختصر به این پرسش ها، تشریح روش های عددی محاسبه مقادیر این توابع  و  ارتباط آنها با جبرخطی عددی نیز بررسی شود.
% ==========+==========+==========+==========+==========
\vspace{4mm}
\\
\noindent
\textbf{واژگان کلیدی:}
توابع ماتریسی. معادلات دیفرانسیل برداری-ماتریسی. معادلات ریکاتی. معادلات سیلوستر. معادلات خطی و غیرخطی برداری-ماتریسی. روش‌های بدست آوردن تجزیه‌ها به صورت عددی. روش شُر. روش نیوتون. شکل طبیعی جردن. انتگرال کوشی.
% ==========+==========+==========+==========+==========
\begin{figure}[h]
\centering
\includegraphics[scale=0.35]{./qr.png}
\caption*{\url{https://meet.google.com/uqs-maiu-zji}}
\end{figure}
% ==========+==========+==========+==========+==========
\end{titlepage}
\end{document}