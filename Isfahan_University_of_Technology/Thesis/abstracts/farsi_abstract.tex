%%%%%%%%%%%%%%%%%%%%%%%%%%%%%%%%%%%%%%%%%%%%%%%%%%%%%%%%%%%%%%%%%%%
%%%%%%%%%%%%%%%%%%%%%%%%%%%%%%%%%%%%%%%%%%%%%%%%%%%%%%%%%%%%%%%%%%%
\vspace*{-1cm}                   %%%  فاصله را خودتان با توجه به حجم چکیده تنظیم کنید  
\thispagestyle{empty}
{\large\bf چکیده: }  
\vspace*{0.1cm}
\\
{\small
امروزه معادلات دیفرانسیل جزئی کسری
(\eng{FPDE})
کاربردهایی در مسائل دنیای واقعی در حوزه‌های علوم و مهندسی پیدا کرده‌اند.
این محبوبیت از ویژگی غیرموضعی مشتق کسری در مقایسه با ماهیت موضعی بودن مشتق مرتبه صحیح نتیجه می‌شود.
یکی از مهم‌ترین
\eng{FPDE}ها،
معادله انتشار کسری-مکانی
(\eng{SFD})
است که در مدل‌سازی انتشار غیرعادی، بررسی پدیده‌های زیرانتشار و توصیف دینامیک‌های آشوبی کاربرد دارد.
معادله
\eng{SFD}
با بعد دو مکانی به‌عنوان یک معادله انتشار کلیدی شناخته می‌شود.
این معادلات از تعمیم مشتقات مکانی مرتبه صحیح به مرتبه کسری در معادله دیفرانسیل مشتقات جزئی حاصل می‌شود.
از آنجا که اکثر معادلات 
\eng{SFD}
به‌صورت تحلیلی قابل حل نیستند، روش‌های عددی مختلفی مانند روش تفاضل متناهی، رویکرد گالرکین ناپیوسته موضعی و روش عناصر متناهی (FEM) برای دستیابی به دقت و کارایی بالا پیشنهاد شده‌اند.
با توجه به این نکته که مشتقات کسری برخلاف مشتقات صحیح غیرموضعی هستند.
در نتیجه صرف نظر از روش گسسته‌سازی مورد استفاده، حجم زیادی از محاسباتی به دلیل غیرموضعی بودن عملگرهای دیفرانسیل کسری مورد نیاز است.
بسیاری از محققان بر روی توسعه الگوریتم‌های سریع برای مقابله با این چالش کار کرده‌اند.
علاوه بر این راه‌حل‌های سریع، رویکردهای محاسباتی موازی مانند کاهش در زمان چندشبکه‌ای
(\eng{MGRIT})
نیز باید به‌عنوان روش‌های مؤثر در نظر گرفته شوند.
در این پایان‌نامه، با بحث در مورد مفاهیم اساسی در آنالیز تابعی، از جمله فضاهای برداری، فضاهای تابع و فضاهای سوبولف و همچنین اصول حسابان کسری شروع می‌کنیم و توضیح می‌دهیم که مشتق‌ها و انتگرال‌های کسری، پایه‌های حسابان کسری هستند و مشتق کسری ریس به‌طور مخصوص برای کاربردهای حوزه مکانی مورد علاقه است.
در ادامه فضاهای کسری مختلف مانند فضاهای سوبولف کسری را بیان کرده و خواص آن‌ها را بررسی می‌کنیم و به فضاهای مرتبط با
\eng{FPDE}ها
نیز اشاره می‌کنیم.
پس از آن، مسئله
\eng{SFD}
را با شرایط مرزی دیریکله بررسی می‌کنیم.
حال
\eng{FEM} 
را بیان کرده و خواص آن را توضیح می‌دهیم.
سپس شکل ضعیف معادله
\eng{SFD}
را می‌سازیم و گسسته‌سازی مکانی-زمانی را با استفاده از گسسته‌سازی مکانی یکنواخت و گسسته‌سازی زمانی غیریکنواخت اعمال می‌کنیم.
این فرایند موجب تولید یک دستگاه بزرگ و تنک از معادلات می‌شود.
برای حل عددی معادله
\eng{SFD}،
روش را به‌عنوان یک حلقه پیشرو زمانی نشان می‌دهیم که در آن یک دستگاه خطی در بعد مکانی در هر مرحله زمانی حل می‌شود.
این حلقه پیشرو زمانی به‌عنوان یک روش تک‌گامی در زمان نیز عمل می‌کند و معادل با حل یک دستگاه بلوکی دوقطری پایین مثلثی در بعد زمان است. 
پس روش‌های مختلفی را برای موازی‌سازی زمانی مورد بحث قرار می‌دهیم و یک تاریخچه مختصر ارائه می‌کنیم.
یکی از روش‌های قابل توجه،
\eng{MGRIT} 
است که از رویکرد کاهش چندشبکه‌ای استفاده می‌کند. 
روش
\eng{MGRIT}
دو مزیت قابل توجه دارد: استقلال نسبی از کدهای موجود و مقیاس‌بندی الگوریتمی بهینه.
در ادامه، از نسخه دوسطحی
\eng{MGRIT}
برای حل آن دستگاه بلوکی دوقطری پایین مثلثی استفاده کرده و به تحلیل عملکرد همگرایی روش می‌پردازیم. 
در نهایت یک مثال عددی را در
\Matlab
و
\XBraid
پیاده‌سازی کرده و عملکرد روش‌های عددی را بررسی می‌کنیم.
نتایج نشان می‌دهد که روش سازگاری و همگرایی کافی را برای حل‌های عددی چنین معادلات
\eng{SFD}
نشان می‌دهد و می‌توان آن را برای حل برخی 
\eng{FPDE}‌های
پیچیده گسترش داد.
}
\\[.1cm]
{\bf رده‌بندی موضوعی:}
\lr{65M55, 65M60, 65Y05}\\[0.1cm]
  {\bf
 واژگان کلیدی:
 }
حسابان کسری،‌ معادلات انتشار کسری-مکانی، روش عناصر متناهی، موازی‌سازی در زمان، روش کاهش در زمان چندشبکه‌ای.