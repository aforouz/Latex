% Chapter 2
% Algorithms and Numerical Methods

% ==================== Algorithms & Numerical Methods ====================

\section{Algorithms \& Numerical Methods}

% ====================Intro=================

\begin{frame}

\frametitle{Algorithms \& Numerical Methods}
1.Algorithms \& Numerical Methods for computing Matrix Square Root.
\newline
\newline
2.Develop a general purpose algorithm for computing $f(A)$
via the Schur decomposition.

\end{frame}

% =============Def_Matrix_Square_Root=======================
\subsection{Matrix Square Root}
\begin{frame}

\frametitle{Definition of Matrix Square Root}
Theorem. (principal square root). Let $A \in C^{n \times n}$ have no eigenvalues on $R{-}$.\newline
There is a unique square root $X$ of $A$ all of its eigenvalues lie in the open right
half-plane,\newline and it is a primary matrix function of $A$. We refer to $X$ as the principal
square root of $A$,\newline and write $X = A^{1/2}$.\newline
\newline
\newline
We will almost exclusively be concerned with the principal square root.
\end{frame}

% ====================listed_methods====================


\begin{frame}

\frametitle{Numerical Methods For computing Matrix Square Root}
1.Schur Method (Bjorck and Hammarling [1983] ).\newline
2.Newton's Method \& It's Variants.\newline
3.Using Square Root Function with other Decompositions

\end{frame}

% ====================Schur_Decomposition===================

\subsection{Schur Method}
\begin{frame}

\frametitle{Schur Decomposition}
\[
A = QTQ^{H}, \quad A \in C^{n \times n} ,
\]
$Q \& Q^{H} $Orthogonal, T Upper Triangular.\newline \newline
Note that the Schur Decomposition is not Unique.\newline \newline
The Schur Decomposition can be computed numerically using the QR-Algorithm.
\end{frame}

% ====================QR-Algorithm=================
\subsection{QR Algorithm}
\begin{frame}
\frametitle{QR Algorithm (J.Francis [1961]).}
This Algorithm is used to find eigenvalues of a matrix using the QR-Decomposition
\[
A_0 =Q_0R_0, \quad A_1 = R_0Q_0, \quad A_1 = Q_1R_1, \quad A_2 = R_1Q_1 \dots
\]
e.g.
\[
   A=  \begin{pmatrix}
        2 & 1 \\
        1 & 2
    \end{pmatrix}, \quad 
    Q = \begin{pmatrix}
        \frac{2}{\sqrt{5}} & \frac{-1}{\sqrt{5}} \\
        \frac{3} {2} & \frac{2}{\sqrt{5}}
    \end{pmatrix}, \quad
    R = \begin{pmatrix}
        \sqrt{5} & \frac{4}{\sqrt{5}} \\
        0 & \frac{3}{\sqrt{5}}
    \end{pmatrix},
\]

\[
A_1 = RQ = 
\begin{pmatrix}
    2.8 & 0.6 \\
    0.6 & 1.2 \\
\end{pmatrix}
\]
The eigenvalues of the matrix are on the main diagonal of $A$.\newline
$\lambda = 1,3.$



\end{frame}

% ====================Computing schur decop via QR Algorithm=======

\begin{frame}

\frametitle{Computing Schur Decomposition using QR-Algorithm.}

\[
U_k = Q_0 Q_1 \dots Q_k .
\]

$as \: A_k \rightarrow T: \quad U^H _K A U_k = A_k \quad \Longrightarrow  \quad A = UTU^H. $\newline \newline

Note that a diagonal matrix is an upper triangular.\newline consequently:

\[
    A=  \begin{pmatrix}
        2 & 1 \\
        1 & 2
    \end{pmatrix},
    T=  \begin{pmatrix}
        3 & 0 \\
        0 & 1
    \end{pmatrix},
    U = \begin{pmatrix}
        \frac {1} {\sqrt {2}} & \frac {-1} {\sqrt {2}} \\
        \frac {1} {\sqrt {2}} & \frac {1} {\sqrt {2}}
    \end{pmatrix}
\]
Off-diagonal enteries of the matrix $A$ converge to $0$.\newline
\end{frame}

% ==================== Schur Method ====================

\begin{frame}

\frametitle{Schur Method\newline (Bjorck and Hammarling [1983] )}
$A \in C^{n \times n}$ is non-singular.\newline
$A = QTQ^H \quad \Longrightarrow \quad f(A) = Qf(T)Q^H ,$\newline \newline
$U^2 = T \quad \Longrightarrow \quad U =? $ ($U$ is upper triangular.)\newline
For diagonal enteries: $(i,i)$
\[
 u_{ii} ^2 = t_{ii},
\]
superdiagonal enteries: $(j>i)$
\[
t_{ii} =\sum_{k=i}^{k=j}u_{ik}u_{kj} \Longrightarrow ,(u_{ii}+u_{jj})u_{ij} = t_{ij} - \sum_{k =i+1}^{j-1}u_{ik}u_{kj}.
\]
First, diagonal enteries are computed then off-diagonal enteries, a super diagonal at a time.The process cannot break down,because $u_{ii}+u_{jj}=0$ is not possible.(why?)
\end{frame}

% ==================== Schur Method ====================

\begin{frame}

\frametitle{Schur Method\newline (Bjorck and Hammarling [1983] )}
\[
X^2 =A , \quad A = QTQ^H ,\quad U^2 =T
\]

\[
\Longrightarrow \quad X =QUQ^H
\]
Summary:\newline
1.Schur Decomposition\newline
2.Computing $u_{ij}$'s.\newline
3.$X = QUQ^H $.\newline \newline
Cost: $\frac {28} {3}n^3 $ flops in total.

\end{frame}

% ====================Newton Method========================

\begin{frame}
\subsection{Newton's Method}
\frametitle{Newton's Method}
$X^2-A =0 , \quad X = ?$\newline
Let $X=Y+E ,$\quad $E$ stands for error part.(to be determined)\newline
$A = (Y+E)^2 = Y^2+YE+EY+E^2 $\newline
\begin{align*}
&
X_0 \ given, \\
&
solve(for \ E) \; X_kE_k+E_kX_k = A-X^2 _k \\
& 
X_{k+1}= X_k + E_k\\
\end{align*}
Lemma. Suppose that in the Newton iteration, $X_0$ commutes with $A$\newline and
all the iterates are well-defined. Then, for all $k$,\newline
$X_k$ commutes with $A$ and $X_{k+1} =\frac {1} {2}(X_k + X_k ^{-1}A)$.
\end{frame}

% ==================== Newton Method_2====================

\begin{frame}

\frametitle{Newton's Method}
The most common choice for $X_0$ is $A$.(or $X_0 = I$ wich generates the same $X_1$ )\newline \newline
The Method is stable if : (Laasonen [1958] ,  Higham [1986]) \[\psi_N := max_{i,j} \frac {1} {2} | 1- \lambda _i ^{\frac {1} {2}} \lambda _j ^{\frac {-1} {2}} | < 1 \]\newline
Many variants of this method exist.\newline \newline
The quadratic convergence of the Newton iteration is shown by Laasonen [1958] \newline
under the assumption that $A$ has real, positive eigenvalues.
\end{frame}

% ====================Comparison=========================

\begin{frame}
    \frametitle{Comparison}
    \begin{table}[]
        \begin{tabular}{ll}
            Iteration & Stable                           \\ \hline
            Newton    & \multicolumn{1}{|l}{(condition)} \\
            DB        & \multicolumn{1}{|l}{Yes}         \\
            CR        & \multicolumn{1}{|l}{Yes}         \\
            In        & \multicolumn{1}{|l}{Yes}        
        \end{tabular}
    \end{table}
    
    \begin{table}[]
        \begin{tabular}{lll}
            Method & Iterations             & Relative Error \\ \hline
            Schur  & \multicolumn{1}{|l}{}  & 5.3e-13        \\
            Newton & \multicolumn{1}{|l}{8} & 5.3e-4        
        \end{tabular}  
        \end{table}
        Moler Matrix :$\psi_N = 10^5 $.
\end{frame}

% ==================== Chapter 2 ====================