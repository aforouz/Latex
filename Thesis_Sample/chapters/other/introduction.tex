\chapter*{پیشگفتار}
\addcontentsline{toc}{chapter}{پیشگفتار}
\thispagestyle{empty}
% ==========+==========+==========+==========+==========
یک دسته از معادلات انتشار پرکاربرد، معادله پخش گرما در بعد دو مکانی است که با تعمیم مشتقات مکانی مرتبه صحیح به مرتبه کسری، معادله انتشار کسری-مکانی
(\eng{SFD})
دوبعدی به دست می‌آید.
از آنجا که بیشتر معادلات
\eng{SFD}
را نمی‌توان به‌صورت دقیق (تحلیلی) حل کرد، روش‌های عددی متفاوت مانند تفاضلات متناهی
\cite{Ref4DXL, Ref9HFCS, Ref14FZLTAL, Ref15CDL}،
عناصر متناهی
\cite{Ref21BTY, Ref24BTWY, Ref25YYNWZL, Ref26ZBZT, Ref28BSYXZ}،
حجم متناهی
\cite{Ref30LZTBA, Ref31JW}
و طیفی
\cite{Ref33SX, Ref34Y}
برای دستیابی به دقت و کارایی بالا به‌طور گسترده مطرح شدند؛ اما به‌علت غیرموضعی بودن عملگرهای دیفرانسیل کسری، کار محاسباتی فشرده‌ای در کنار نوع گسسته‌سازی اعمال‌شده مورد نیاز است
\cite{Ref42GBTJL}.
به همین دلیل بسیاری از پژوهشگران سعی کرده‌اند الگوریتم‌های سریع مناسبی را برای مقابله با این چالش شناسایی کنند
\cite{Ref43PS, Ref48CLZ}.
به‌علاوه می‌توان از محاسبات موازی به‌عنوان یک راهبرد اساسی استفاده کرد.
گونگ\efn{Gong}
و همکاران الگوریتم‌های موازی بر پایه پردازنده و گرافیک رایانه را برای معادلات
\eng{SFD}
يک‌‌بعدی با مشتق کسری
ریس\efn{Riesz}
ارائه کردند
\cite{Ref49GBT, Ref50WLGZX}
که افزایش سرعت‌ آن با موازی‌سازی مکانی همراه با رویکرد گام‌های زمانی متوالی، با استفاده از انتشاردهنده زمان برای ادغام از یک زمان به زمان دیگر به ‌دست‌ می‌آید.
با این ‌حال برای افزایش سرعت محاسبات در آینده، باید به تعداد پردازنده‌های بیشتر به‌جای پردازنده‌های سریع‌تر تکیه کرد.
پس می‌توان نتیجه گرفت که الگوریتم‌های حل‌کننده محدود به موازی‌سازی مکانی برای مسائل دارای رفتار تکاملی، مستلزم زمان محاسباتی طولانی هستند و اغلب برای حل معادلات
\eng{SFD}
چندبعدی نیازمند منابع محاسباتی فراتر از منابع موجود هستند.
\\
اگرچه زمان به‌طور ذاتی متوالی است اما فن ﻣﻮﺍﺯﯼﺳﺎﺯﯼ ﺩﺭ ﺯﻣﺎﻥ، یک روش ﺟﺪﯾﺪ نیست و الگوریتم‌های دارای این فن، از تقاضای بالایی در ده سال اخیر برخوردار بوده‌اند.
در حال حاضر الگوریتم‌های فراواقعی در زمان
\cite{Ref51LMT}
و کاهش در زمان چندشبکه‌ای
(\eng{MGRIT})
\cite{Ref52FFKMS}
دو مورد مؤثر هستند.
وو\efn{Wu}
و
ژو\efn{Zhou}
خواص همگرایی الگوریتم فراواقعی را برای معادلات
\eng{SFD}
از طریق معادلات دیفرانسیل معمولی در گام‌ زمانی ثابت تجزیه و تحلیل کردند
\cite{Ref54WZ}،
اما آزمایشات در مقیاس بزرگ ارائه نشده‌ است.
با توجه به‌ این‌ که الگوریتم فراواقعی را می‌توان به‌عنوان نسخه دوسطحی روش کاهش چندشبکه‌ای تفسیر کرد
\cite{Ref52FFKMS, Ref55GV}،
همروندسازی آن به دلیل حل شبکه درشت متوالی بزرگ هنوز محدود است.
\\
یکی از مراجع مهمی که در تهیه این پایان‌نامه از آن استفاده شده، مقاله یو و همکاران
\cite{Ref0YSXBP}
است.
اکثر مطالب این پایان‌نامه بر سه موضوع حسابان کسری، روش عناصر متناهی و روش کاهش در زمان چندشبکه‌ای تمرکز دارند.
در فصل اول به مرور مفاهیم مقدماتی از آنالیز تابعی و آشنایی کافی با حسابان کسری پرداخته شده است.
همچنین فضاهای کسری مورد نیاز همراه با ویژگی‌های مهم آنان در این فصل بررسی شده‌اند.
سپس مسئله
\eng{SFD}
مورد نظر پایان‌نامه در شروع فصل دو بیان شده و پس از بازگویی
\eng{FEM}،
از آن جهت گسسته‌سازی مسئله
\eng{SFD}
مورد نظر و ساخت یک حلقه پیشرو زمانی استفاده شده است.
در نهایت جهت سرعت بخشیدن به محاسبه حلقه پیشرو زمانی به‌دست‌آمده، روش
\eng{MGRIT}
بیان و آنالیز همگرایی نسخه دوسطحی آن بررسی شده است.
روش‌های عددی نیز با استفاده از نرم‌افزار
\Matlab
و بسته نرم‌افزاری
\XBraid
پیاده‌سازی شده که کدهای مربوط به پیاده‌سازی در پیوند زیر قابل مشاهده هستند.
\begin{center}
\href{https://github.com/aforouz/Research_Papers/tree/main/MSc_Thesis}{\cod{https://github.com/aforouz/Research_Papers/tree/main/MSc_Thesis}}
\end{center}
در این پایان‌نامه، همه توابع حقیقی مقدار هستند.
همچنین ناحیه مستطیلی
$\Omega = (a, b) \times (c, d)$
که
$a,b,c,d \in \Real$،
به‌عنوان دامنه مکانی دوبعدی در نظر گرفته شده و تمامی تعریف‌ها براساس آن ارائه شده‌اند.