\chapter{
مقدمه
}
\label{chp:chap1}\thispagestyle{empty}

در این فصل، پس از بیان مفاهیم مقدماتی از آنالیز تابعی، حسابان کسری همراه با انتگرال و مشتق کسری مرور می‌شود.
انتگرال کسری ابزار مهمی از حسابان کسری است که مشتق کسری براساس آن فرمول‌بندی می‌شود.
سپس فضاهای کسری مورد نیاز بیان شده‌اند و در نهایت توضیح مختصری از معادلات با مشتقات کسری و اهمیت آنان ارائه شده است.
% ==========+==========+==========+==========+==========+==========+==========+==========
\section{
آنالیز تابعی
}
آنالیز تابعی به مطالعه توابع و فضاهای تابعی می‌پردازد.
همچنین به‌عنوان یک موضوع متحدکننده در نظر گرفته می‌شود که بسیاری از مفاهیم جبرخطی و آنالیز حقیقی (یا مختلط) را با تأکید بر فضاهای با بعد نامتناهی تعمیم می‌دهد.
در این بخش به ارائه مختصری از مفاهیم موردنیاز آنالیز تابعی شامل فضاهای برداری، فضای توابع و فضای
سوبولف\efn{Sobolev}
پرداخته شده است.
مطالب این بخش برگرفته از مرجع
\cite{Ref111R}
هستند.
% ==========+==========+==========+==========+==========
\subsection{
فضاهای برداری
}

\begin{definition}[فضای نرم‌دار]
\label{normDefinition}
فضای برداری حقیقی
$X$
را در نظر می‌گیریم.
تابع
$\normS{X}{ \cdot } : X \to \RealPZ$
یک نرم روی
$X$
است هرگاه برای هر بردار
$u,v \in X$
و اسکالر
$c \in \Real$،
شرایط زیر برقرار باشند
\begin{itemize}
\item
$\normS{X}{c u} = \abs{c} \normS{X}{u}$،
\item
$\normS{X}{u + v} \leq \normS{X}{u} + \normS{X}{v}$،
\item
$\normS{X}{u} = 0 \Leftrightarrow u = 0$.
\end{itemize}
فضای برداری
$X$
همراه با نرم
$\normS{X}{\cdot}$
یک فضای نرم‌دار است.

\end{definition}

\begin{note}
اگر در شرایط تعریف
\ref{normDefinition}،
$\normS{X}{u} = 0$
نتیجه ندهد
$u = 0$،
آن‌گاه تابع
$\normS{X}{\cdot}$
یک نیم‌نرم بوده و با
$\absS{X}{\cdot}$
نمایش داده می‌‌شود.
\end{note}

\begin{definition}[هم‌ارزی نرم‌ها]
فرض کنیم دو نرم
$\normS{A}{\cdot}$
و
$\normS{B}{\cdot}$
بر فضای برداری
$X$
تعریف شده باشند.
این دو نرم هم‌ارز هستند هرگاه ثابت‌های
$c_{1}, c_{2} \in \RealP$
چنان موجود باشند که برای هر
$u \in X$،
رابطه زیر برقرار باشد
\begin{equation*}
c_{1} \normS{A}{u} \leq \normS{B}{u} \leq c_{2} \normS{A}{u}
\sdn
\end{equation*}
\end{definition}

\begin{definition}[فضای ضرب داخلی]
فضای برداری حقیقی
$X$
را در نظر می‌گیریم.
تابع
$\innerS{X}{\cdot}{\cdot} :‌ X \times X \to \Real$
یک ضرب داخلی روی
$X$
است هرگاه برای هر بردار
$u,v,w \in X$
و اسکالر
$c \in \Real$،
شرایط زیر برقرار باشند
\begin{itemize}
\item
$\innerS{X}{u}{u} \geq 0$،
\item
$\innerS{X}{u}{v} = \innerS{X}{v}{u}$،
\item
$\innerS{X}{c u + v}{w} = c \innerS{X}{u}{w} + \innerS{X}{v}{w}$،
\item
$\innerS{X}{u}{u} = 0 \Leftrightarrow u = 0$.
\end{itemize}
فضای
$X$
با ضرب داخلی
$\innerS{X}{\cdot}{\cdot}$
یک فضای ضرب داخلی نامیده می‌شود.
\end{definition}

\begin{note}
برای هر فضای ضرب داخلی
$X$،
می‌توان یک نرم القایی از ضرب داخلی به‌صورت زیر تعریف کرد
\begin{equation*}
\normS{X}{u} = \sqrt{\innerS{X}{u}{u}}
\sqq
u \in X
\sdn
\end{equation*}
پس هر فضای ضرب داخلی،‌ یک فضای نرم‌دار است.
\end{note}

\begin{theorem}[نامساوی کوشی-شوارتز]
 برای هر
$u,v \in X$
در فضای ضرب داخلی
$X$،
نامساوی زیر برقرار است
\begin{equation*}
\abs{ \innerS{X}{u}{v} } \leq \sqrt{ \innerS{X}{u}{u} } \sqrt{ \innerS{X}{v}{v} }
\sdn
\end{equation*}
\end{theorem}

% ==========+==========+==========+==========+==========+==========+==========+==========